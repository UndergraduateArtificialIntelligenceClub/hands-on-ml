
\documentclass[xcolor=dvipsnames, fontsize=11pt, % Font size
pagesize, % Write page size to dvi or pdf
parskip=half-, t]{beamer}

\input{beamer.tex}
\usepackage{multicol}
%\titlegraphic{\includegraphics[height=0.15\textwidth]{../logo.png}}
\title[Artificial Intelligence for Beginners]{Gradient Descent}
\subtitle{Towards Neural Networks}
\author[Justin Stevens]{\large Justin Stevens} % Your name
\date{}
\setbeamertemplate{button}{\tikz
	\node[
	inner xsep=10pt,
	draw=structure!80,
	fill=structure!50,
	rounded corners=4pt]  {\large\insertbuttontext};}
\usepackage{asymptote}
\usepackage{animate}
\usepackage{xmpmulti}
\begin{document}
	\renewcommand{\thefootnote}{\fnsymbol{footnote}}
	\begin{frame}[c]
	\centering
	\titlepage
\end{frame}
\section{Classifying Digits through MNIST}
\subsection{Defining the Problem}
\begin{frame}[c]{Example Images}
\begin{figure} 
\center
\includegraphics{mnist_100_digits.png}
\caption{How would you devise a system for a \textbf{computer} to classify the digits? What assumptions do we have to make about the data set, known as MNIST? }
\end{figure}
\end{frame}

\begin{frame}[c]{Assumptions}
\begin{itemize}
\item The MNIST database contains thousands of handwritten digits. 
\item Each data-point contains both an image, and the desired digit.
\item The images are $28\times 28$ pixel arrays, with each pixel ranging from $0$ to $255$. 
\item $60,000$ of these are designated for training purposes, and $10,000$ for test purposes.
\item We'll build a model from these, that will learn to classify digits!
\end{itemize}
\end{frame}

\begin{frame}{What we're building towards}
\begin{figure}
\center
\includegraphics[scale=0.36]{goal.png}
\caption{A simple neural network structure. The input vectors on the left hand side have $28\times 28=784$ inputs for each pixel, and the output layer has $10$ digits, one for each number from $0$ to $9$.}
\end{figure}
\end{frame}



\subsection{References}
\begin{frame}[c]{References}
\setbeamertemplate{itemize items}[triangle]

\href{http://www.slader.com/textbook/9780495011668-stewart-calculus-early-transcendentals-6th-edition/}{\beamergotobutton{Stewart Calculus: Early Transcedentals, 6th Edition}} \smallskip

\href{https://www.youtube.com/playlist?list=PLDesaqWTN6ESk16YRmzuJ8f6-rnuy0Ry7}{\beamergotobutton{Professor Leonard Calculus 3 (Full Length Videos)}} \smallskip

\href{http://tutorial.math.lamar.edu/Classes/CalcIII/CalcIII.aspx}{\beamergotobutton{Paul's Online Math Notes, Calculus III}} 
\end{frame}
\end{document}